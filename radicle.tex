\documentclass[a4paper, oneside, 9pt, draft]{amsart}
\usepackage[english]{babel}
\usepackage{amsmath}
\usepackage{amssymb}
\usepackage{amsfonts}
\usepackage{amsaddr}
\usepackage{mathtools}
\usepackage{diagbox}
\usepackage{booktabs}
\usepackage{enumitem}
\usepackage{color}
\usepackage{hyperref}
\usepackage{tikz}
\usepackage{accents}
\usepackage{standalone}
\usepackage{bm}
\usepackage{multicol}
\usepackage{stmaryrd}
\usepackage{graphicx}
\usepackage{float}
\usepackage{syntax}
\usepackage[final]{listings}
\usepackage{textcomp}
\usepackage[font={small,sc}]{caption}
\usepackage{fancyvrb}

\usetikzlibrary{fit}
\usetikzlibrary{backgrounds}
\usetikzlibrary{positioning}
\usetikzlibrary{decorations}
\usetikzlibrary{decorations.pathmorphing}
\usetikzlibrary{arrows.meta}
\usetikzlibrary{chains}
\usetikzlibrary{shapes.multipart}
\usetikzlibrary{scopes}
\usetikzlibrary{arrows}

\hypersetup{colorlinks,
    citecolor=black,
    filecolor=black,
    linkcolor=black,
    urlcolor=blue,
    pdftex}

\setlist[description]{leftmargin=1.2cm, labelindent=\parindent}
\setlist[enumerate]{leftmargin=1.2cm, labelindent=\parindent}

\lstset{
    language=Scheme,
    numbers=left,
    basewidth=0.5em,
    xleftmargin=5.0ex
}

\newcommand*\eg{e.g.\ }
\newcommand*\ie{i.e.\ }

\title[Radicle]{Radicle \\ {\tiny Draft 1.0}}
\author{\Small Julian K. Arni}
\address{\normalfont \texttt{julian@monadic.xyz}}
\author{\Small James Haydon}
\address{\normalfont \texttt{james@monadic.xyz}}

\date{September 2018}

\newcommand{\oscoin}{\textsc{\small{oscoin}}}
\newcommand{\mathsc}[1]{\text{\normalfont\scshape#1}}
\newcommand{\cvdots}{\multicolumn{1}{c}{\small{\vdots}}}
\newcommand{\notice}[1]{{\color{red}#1}}
\newcommand{\comment}[1]{{\color{gray}#1}}
\newcommand{\todo}[1]{\comment{\noindent\texttt{\small\emph{TODO: #1}}}}

\newcommand{\denotation}[2]{
    \ensuremath{\mathcal{E}_{#1} \llbracket #2 \rrbracket }}
\newcommand{\denotationdesc}[3]{
    \ensuremath{\mathcal{E}_{#1} \llbracket #2 \rrbracket \env \cont \mem &= #3  }}
\newcommand{\variable}{\nu}
\newcommand{\prog}{\pi}
\newcommand{\rad}{\textsc{\small{Radicle}}}
\newcommand{\rads}{\textsc{\footnotesize{Radicle}}}
\newcommand{\lam}{\text{\texttt{lambda }}}
\newcommand{\define}{\text{\texttt{define}}}
\newcommand{\deref}{\text{\texttt{deref!}}}
\newcommand{\mkref}{\text{\texttt{ref!}}}
\newcommand{\writeref}{\text{\texttt{write-ref!}}}
\newcommand{\quotem}{\text{\texttt{quote}}}
\newcommand{\evalm}{\text{\texttt{eval}}}
\newcommand{\ifm}{\text{\texttt{if}}}
\newcommand{\stdfn}[2]{
    \texttt{#1}
    #2
    }

\def\S{\mathcal{S}}
\def\B{\mathcal{B}}

\def\Ident{\mathrm{Ident}}
\def\String{\mathrm{String}}
\def\Func{\mathrm{Func}}
\def\Bool{\mathrm{Bool}}
\def\Num{\mathrm{Num}}
\def\Sym{\mathrm{Sym}}
\def\Keyword{\mathrm{Keyword}}
\def\String{\mathrm{String}}
\def\List{\mathrm{List}}
\def\Dict{\mathrm{Dict}}
\def\Ref{\mathrm{Ref}}
\def\evalAddr{\text{\texttt{eval-addr}}}
\def\ikeyword{\mathrm{keyword}}
\def\iatom{\mathrm{atom}}
\def\inum{\mathrm{num}}
\def\ibool{\mathrm{bool}}
\def\istring{\mathrm{string}}
\def\ilist{\mathrm{list}}
\def\idict{\mathrm{dict}}
\def\iref{\mathrm{ref}}
\def\ifunc{\mathrm{func}}

\DeclareMathOperator{\eval}{\Xi}

% Adjust spacing around equations. It can be a bit tight!
\expandafter\def\expandafter\normalsize\expandafter{%
    \normalsize%
    \addtolength{\abovedisplayskip}{2pt}%
    \addtolength{\abovedisplayshortskip}{2pt}%
    \addtolength{\belowdisplayskip}{2pt}%
    \addtolength{\belowdisplayshortskip}{2pt}%
}%

\newenvironment{fig}
  {\par\bigskip\noindent\minipage{\linewidth}}
  {\endminipage\par\bigskip}

% Customize Verbatim environment.
\RecustomVerbatimEnvironment
    {Verbatim}{Verbatim}
    {framesep=2mm, frame=single, label=REPL, rulecolor=\color{black}, fontsize=\small}

% No paragraph indentation after section headers.
\makeatletter
\let\@afterindenttrue\@afterindentfalse
\makeatother

% Tikz stuff

\tikzstyle{block} = [rectangle, draw,
    text width=5em, text centered, rounded corners, minimum height=4em]
\tikzstyle{line} = [draw, -latex']

\setlength{\textwidth}{\paperwidth}
\addtolength{\textwidth}{-4cm}
\setlength{\textheight}{\paperheight}
\addtolength{\textheight}{-5cm}
\calclayout

\begin{document}
\begin{abstract}
    Replicated state machines are employed in a wide variety of applications
    requiring distributed coordination. Because RSMs must always adhere to
    particular semantic conditions, foremost determinism, rather than reprogram
    their semantics in every new instance, we propose a language, \rads{}, which
    substantially simplifies the process of defining such state machines. As
    \rads{} may be made into any RSM via \emph{eval-redefinition}, it can be
    considered a universal replicated state machine.

    This mechanism also allows changing the semantics of running state machines
    without loss of state, and with the same consensus guarantees as provided
    by the underlying consensus system.
\end{abstract}
\maketitle

\setlength{\columnsep}{20pt}
\begin{multicols}{2}

\section{Introduction}
\label{s:introduction}

% Replicated state machines. Examples (Paxos, blockchains, etc.). It's purpose
% and constraints

% Variety of applications. Is there a unifying abstraction? Needs to be
% deterministic, but aside from that, a DSL-language.

% Mention radicle.

% Allows for updates that benefit from the same consistency guarantees as the
% operations themselves. Helps solve the problem of software updates in Paxos
% etc, and a more powerful form of self-amendment in blockchains. [Note that
% perfomance gains in the compiler/interpreter must happen differently, though
% separating changes that can be semantic from those that are not helps.]

% Effect system. Separating out impurity. In the case of blockchains, clients
% are *also* programmable, without sacrificing consensus.

% outline.
% Section 2) Some examples: an upgradable key-value store. A version of nomic.
% Section 3) Formal description of language.

Replicated state machines are a widely used paradigm to program fault-tolerant
systems. The paradigm involves deploying a deterministic state machine across
multiple nodes (servers); these nodes can respond to client requests, and by
agreeing on the order of these requests (consensus), ensure agreement on their
state and output. If some fraction of nodes is unavailable (or in the looser
requirement of byzantine fault tolerance, if they are responding in arbitrary
ways), the overall system can
still function correctly.

The service these systems replicate may be key-value stores, file-systems,
append-only logs, account balances, etc. Each of these
services is often re-implemented anew, leading to substantial development
costs, as well as subtle bugs in the interim state of the system during the
inevitable software upgrades. In this paper, we describe a language, \rad{}, for
defining the behaviour of replicated state machines (RSMs) independently of the
underlying consensus. The language is designed so that new domain-specific
languages (DSLs) can easily be defined for each service provided by the system.
Each such DSL is the definition of an RSM. The expressions of the DSL
are the RSMs inputs; the value such an expression evaluates to its outputs; and
the changes in the environment its state changes.

Thus, if for example the RSM we are interested
in is a ledger of accounts, expressions or inputs may be transfers, the
state may be the balance and ownership of accounts, and outputs may be the new
balances of affected accounts or an error message if the transfers are not
allowed (due to insufficient funds or incorrect permissions).

Additionally, using the same mechanism as for DSL-definition (namely, an
\emph{eval redefinition}), we provide a way for upgrading the DSL itself
\emph{with the same guarantees of agreement} between nodes as the underlying
consensus, reducing the coordination difficulty of an upgrade. DSLs can be
defined in such a way that their \emph{re}-definitions are themselves just one of the
sorts of inputs RSM accepts; thus nodes will agree on the ordering of the
upgrade with respect to other inputs, and will not go out of sync as a
consequence of update.

While (crucially for the purpose of ensuring consensus) the core of the language
is deterministic, \rad{} also possesses an additional set of primitive operations
that allow side-effects. We show how a publish-subscribe model for side-effects
(or in the formalism of \cite{Cartwright1994}, an effect-handling central which
is never provided continuations) allows responding to outputs of the state
machine or state changes in a non-deterministic way, without endangering the
determism of the state machine itself. This in turn makes the separation between
reads and writes correct by construction.

\rad{} has been developed in the context of a broader effort to create a
community-owned platform for open-source development (\oscoin{}), which
includes a consensus algorithm and a networking component. The \oscoin{}
platform allows users and communities to create permissioned and permissionless
RSMs with their own semantics, be it to maintain decentralized
version-controled code, issues, pull requests, or collective decision-making.
\rad{} is
oriented towards making that process as simple and clear as possible. We do
not in this paper further discuss the broader \oscoin{} system.

In the rest of the paper, we describe the language in more detail (Section
\ref{s:language}) and show some example applications built with \rad{}---first,
an upgradable key-value store and then a currency
(Section \ref{s:examples}).

\section{Language Syntax and Semantics}
\label{s:language}

While the individual parts of \rad{} do not bring anything
new to programming language theory, its intended domain of use
differs significantly from that of other programming languages and the set of
design choices that characterize \rad{} make for a unique language.

\rad{} provides a way to program the behaviour of RSMs. A particular instance of
a \rad{} RSM is an append-only sequence of \rad{} inputs, which grows over time,
is managed by some form of \emph{consensus} (Paxos, Raft, proof-of-work,
etc.), and which is valid according to the \rad{} semantics (see section
\ref{semantics-spec}). That is, the inputs are accepted in order and without
error. We'll refer to such an RSM as a \emph{\rad{} RSM}, but because we usually
think of \rad{} as being used on a blockchain, or e.g. on a
Scuttlebutt feed (unforgeable, single-owner, append-only logs), we'll also refer
to such RSMs simply as \emph{chains}. In such a context it is still useful to
run some computations through the \rad{} interpreter locally, that is, evaluate
some inputs speculatively without submitting them to the underlying consensus
mechanism. Such computations (and any other activity not regulated by the
consensus protocol) are referred to as \emph{off-chain}, while those of
the RSM that are under consensus are referred to as \emph{on-chain}.

\rad{} strives:
\begin{itemize}
\item[(1)] To be deterministic, so that \rad{} programs specifiy deterministic
  state machines.
\item[(2)] To be powerful enough to be a `universal state-machine'.
\item[(3)] To be able to restrict this power appropriately, so that malicious
  inputs can be rejected, and to aid reasoning about the behaviour of valid
  inputs.
\item[(4)] To be concise, expressive and emphasize correctness, so that new
  chains, with new semantics can be created cheaply (in terms of development
  time), be easily understood by all participants, and hopefully bug-free.
\item[(5)] To work well in a collaborative setting, in which code is submitted
  by multiple parties, in a potentially unspecified order.
% TODO: point (5) is unconvincing
\item[(6)] To have the simplest possible underlying semantics, so
  that \rad{} interpreters are well specified, and so that \rad{} can be used in
  security-sensitive situations.
\item[(7)] To have the ability to interpret itself, so that the semantics of a
  chain may be modified according to its own semantics---that is, on-chain.
\end{itemize}

The design choices we made with \rad{} are:
\begin{itemize}
\item[(a)] It is a high-level, homoiconic LISP dialect.
\item[(b)] It has the ability to redefine a special function \texttt{eval}, which
  is used as part of its own evaluation.
\item[(c)] It has first-class functions.
\item[(d)] It is dynamically typed.
\item[(e)] It is lexically scoped with a \emph{hyperstatic} global environment.
\item[(f)] It only has immutable values.
\item[(g)] It has a deterministic effect system for managing state (references).
\end{itemize}

\begin{figure}[H]
\small
\begin{tabular}{l|ccccccc}
    & a & b & c & d & e & f & g\\
  \hline
  1 &   &   &   &   &   &   & + \\
  2 & + &   &   &   &   &   & \\
  3 &   & +\\
  4 & + &   & + & - &   & +\\
  5 & \\
  6 & + &   &   &   & + & +\\
  7 & + & +\\
\end{tabular}
\normalsize
\caption{Summary of design choices and how they interact with \rad{}'s design goals.}
\end{figure}

LISP is well-known for being a high-level family of languages having very concise
self-interpreters, and the Scheme specification has boiled down the
semantics to a small core. Choosing a LISP helps with (2), (4) and (7), and, in
particular, basing the design on Scheme helps with (6).

The redefinition of \texttt{eval} is what enables (3).

Languages with first-class functions are particularly expressive, so (c) helps
with (4).


Point (d) is a compromise: it is hard to satisfy the other points (especially
(7)) with strong static types, even though this would help with (4). Although 
recent research indicates that it is possible to add type systems to even 
strongly-normalizing languages with self-interpreters (see for example
\cite{Brown2016}), this constraint would also inhibit the ability to compile
other languages into \rad{}.

The \emph{hyperstatic} global environment means that the resolution of free
variables takes place at the definition site rather than call-site. See Section
\ref{s:hyperstatic} for explanations for how this helps with (6).

The language is kept as pure and immutable as possible, which limits
expressivity but emphasizes clarity and correctness (goal (4)). The abundance of
mutability would also probably hinder (6).

Since RSMs are inherently stateful, we felt that the inclusion of a reference
type for managing state would help in the specification of semantics, however we
will discuss the alternatives.

The remainder of this section expands upon these goals and
design choices.


\subsection{LISP}

Scheme has the particularity that a minimal interpreter of itself can
be implemented in a few lines of code. This is achieved by:
\begin{itemize}
  \item Code being represented by the core data-structures of the language, so
      that manipulating code is as straighforward as any data manipulation.
  \item The language being built up by a very small set of primitives: a few
    special forms and some primitive functions.
\end{itemize}
Let's consider a basic example of a chain which might want to change a part of
its semantics. In this chain, participants may want to \texttt{boot} other
participants if some condition is met, which is defined by a predicate
\texttt{bootable?}. For example, \texttt{bootable?} might be defined as:
\begin{lstlisting}
(def bootable?
  (fn [a b]
    (and (admin? a) (not (admin? b)))))
\end{lstlisting}
That is, admins may boot non-admins. At some point the participants may decide
that this is too crude, and that a participant should only be booted if the
majority of participants agree. For this change to take place, most likely one of
the inputs to the system will contain code for a new version of
\texttt{bootable?} (the semantics of when such a message is to be accepted would
probably be defined by yet another predicate). This code must be transmited and
interpreted in the simplest and most transparent way possible. By choosing a LISP,
this code is represented by simple data-structures, and the interpretation process is
very direct (simplifying parsing and abstract syntax trees, etc.). This
gives the participants more confidence in their understanding of the current and
proposed semantics.

\subsection{Data types}
\rad{}'s data types are: booleans (\texttt{\#t} and
\texttt{\#f}); strings (a sequence of characters within double quotes, with
\verb$\$ as the escape character), symbols, keywords, lists, dictionaries
(maps), vectors, and numbers (currently only arbitrary-precision decimals). These
datatypes are all immutable; additionally \rad{} supports \emph{refs}---mutable
references that can hold any other datatype.

For more details on radicle values, see Appendix \ref{value-definition}.

\subsection{Hyperstatic Environments}
\label{s:hyperstatic}

Most scripting languages are intended for programs where the developer can
decide exactly where code should be placed, and be fully aware of the context
that precedes it. One can always add a line \texttt{y = bar(x)} immediately
\emph{after} \texttt{x = foo(3)}, and not worry if something else was
added in between that changes what \texttt{x} refers to (excepting pathological
cases of distributed code collaboration).

This is not true in the environment in which \rad{} is intended to operate. In this
environment, users submit individual expressions or declarations to a running
system that is \emph{at the same time} also accepting inputs from other users.
Moreover, the view of the program that the submitter has at the time of
submission may \emph{already} be outdated. This aspect is even more significant
in the context of a blockchain \cite{Nakamoto2008}, where expressions are
queued until a block is created by a node, but the expressions in one queue are not
necessarily accessible to other nodes.

This fact may lead to bugs or attacks. Consider the following snippet:
\begin{lstlisting}
(def transfer (fn [from to amount] ...))
(def account-alice ...)
(def account-bob ...)
(def transfer-to-alice (fn [from amount]
  (transfer from account-alice amount))
\end{lstlisting}
A user now wants to transfer some amount to Alice. However, in a language such
as Scheme the function call \texttt{(transfer-to-alice account-bob 10)} may
behave differently than this snippet alone indicates in two ways. First,
\texttt{transfer-to-alice} and \texttt{account-bob}, which explicitly appear in
the expression being submitted, may be shadowed by a new definition. Second, the
variables \texttt{transfer} and \texttt{account-alice} used in the body of
\texttt{transfer-to-alice}, and \emph{any other} free variables used in the body
of \texttt{transfer} or transitively, in the function bodies of the functions
\texttt{transfer} calls, may have been redefined. The creation of a function via
a lambda delays two things at once: execution of the body and resolution of free
variables in the body.

One solution is to not allow the redefinition of variables at all. This is
indeed what some smart contract \cite{Szabo94} languages do, and can be implemented easily in
\rad{} as well (see Section \ref{s:reflective-towers}). But this comes with its
own problems; one cannot define an improved version of a function that shadows
the old one, and must instead accept a more and more polluted environment (which
additionally may impact memory usage).

An alternative is to implement a mechanism (call it \texttt{protect-at-line}) to
make expressions invalid if any of the variables they rely on have been
redefined since a specified line of code.
However this option is quite severe. Consider:
\begin{lstlisting}
(def x 3)
(def foo (fn [] x))
(def x 5)
\end{lstlisting}
The function call \texttt{(protect-at-line 2 (foo))} will fail despite having a
perfectly reasonable interpretation---namely, \texttt{(foo)} \emph{with}
\texttt{x} \emph{as if used after line 2} (i.e., with \texttt{x} being
\texttt{3}). For more complex expressions, which may rely on many variables,
this problem becomes more significant.

This reasonable interpretation is in fact precisely what hyperstatic
environments provide. A function call \texttt{(foo)} after line 3 would, in this
model, still evaluate to \texttt{3}. The free variables of a function refer to
the values of variables \emph{when they were defined}. This is the semantics
\rad{} adopts in general (and can be found in a few other languages, such as
Forth and Standard ML\cite{SML1997}).

Now \texttt{(protect-at-line (foo))} may be given a much simpler definition:
\texttt{(foo)} if \texttt{foo} \emph{alone} has not been redefined, and otherwise
an exception. If \texttt{x} has been redefined, this is no longer of concern, as this
redefinition does not change the meaning of \texttt{foo}.

%%% End section

\subsection{\texttt{eval} redefinition}

\rad{} should be able to emulate any state machine. Of course, most state-machines
don't accept arbitrary inputs, and certainly not inputs which change the
semantics of the state-machine. Thus \rad{} should have some built in mechanism
for controlling how inputs are interpreted. To begin, inputs are sent
directly for evaluation by the `base' evaluator, so it is natural to associate
modifying this process with redefining a function which represents evaluation.
When taken to the extreme, \emph{all} evaluation is done with a call to the
special function \texttt{eval}, and this creates a reflective tower of
interpreters. In \rad{}, a more limited form of \texttt{eval}-redefinition is
available. As an example:
\bigskip
\begin{Verbatim}
> (write-ref eval-ref (fn [expr] 3))
=> ()
> 5
=> 3
\end{Verbatim}
Contrast this with the sort of \texttt{eval}-redefinition that is available in
the language Black\cite{Asai1997}:
\begin{lstlisting}
(exec-at-metalevel
  (let ((old-eval base-eval))
    (set! base-eval
          (fn [exp env cont]
            (write exp) (newline) (old-eval exp env cont)))))
\end{lstlisting}
After this the Black REPL will behave as follows:
\bigskip
\begin{Verbatim}
> (+ 1 2)
(+ 1 2)
+
1
2
=> 3
\end{Verbatim}
Thus, the new evaluation function is used recursively, at all levels of the
expression \texttt{(+ 1 2)}.

Compare this to the following \rad{} session:
\bigskip
\begin{Verbatim}
> (def old (read-ref eval-ref))
> (def new (fn [e] (do (print! e) (old e))))
> (write-ref eval-ref new)
=> ()
> (+ 1 2)
(+ 1.0 2.0)
=> 3.0
\end{Verbatim}
In Black, \texttt{old-eval} calls out to
\texttt{base-eval} and, upon subsequent evaluation, this will refer to the new
definition. This is what makes the new evalution behaviour take effect at all
levels of evaluation. In \rad{}, the new evaluation is only invoked at the
topmost level, on new inputs. If one wants to make the evaluation behaviour
recursive this must be coded explicitely.

In most cases a \rad{} chain will define a domain-specific language to be
interpreted in some narrow way, with only some calls to the base eval for
reifying function definitions in case one wants to modify the behaviour of the
state-machine. Furthermore, this form of evaluation redefinition doesn't (by
default) create a tower of interpreters. Evaluating in a tower requires advanced
interpretation methods to avoid performance overheads, and this is all the more
important in \rad{}'s case where the chains may be very long-lived. Such methods
are still the subject of active reasearch (see for example \cite{Asai2014},
\cite{Thyer1999}, \cite{Amin2017}, \cite{Brown2017}), and may interfere with
other requirements of the language. Moreover, since eval-redefinition affects
all sub-expressions, and may affect not just the ``immediate'' interpreter, but
ones in which it is defined, reasoning about eval-redefinitions in towers can be
very difficult. For these reasons we believe that this limited form of
eval-redefinition suits the needs of \rad{} better than the degree of reflection
that can be found in languages such as Black.

%% \rad does not have macros. Instead, it allows for a complete redefinition of
%% the intepreter via redefinitions of \texttt{eval}.

%% \begin{verbatim}
%% (def eval (fn [expr] 3)) ;;
%% 5 ;; => 3
%% \end{verbatim}

%% This enables users to very easily define sublanguages for new chains, or amend
%% the language running in the current one.

For a more involved example, consider a simplistic key-value store:
\input{out/kv0.rad-tex}
After which we have:
\bigskip
\begin{Verbatim}
> (set key1 3)
=> ()
> (get key1)
=> 3
> (+ 3 2)
=> 'invalid-command
\end{Verbatim}
Note that nested expressions are not evaluated:
\medskip
\begin{Verbatim}
> (set key2 (+ 3 2))
=> ()
> (get key2)
=> '(+ 3 2)
\end{Verbatim}

% TODO: maybe move to Related Work section
\subsection{Connection to Reflective Towers}
\label{s:reflective-towers}

The \texttt{eval}-redefinition mechanism resembles prior work on
\emph{reflective towers}. A reflective tower is an infinite sequence of
interpreters (called `levels') $L_0$, $L_1$, ..., where level $L_n$ is
interpreted by $L_{n+1}$. Reflective towers allow both \emph{reification}---the
ability to inspect a computation via constructs of a higher level---and
\emph{reflection}, that is, the ability to define and enter new, lower levels.
Conceptually, \rad{} differs from reflective towers by only allowing reflection.
Thus, the only levels that exist are the ones programs create. Queinnec has
quipped that reflective languages ``plunge us into a world with few laws, and
hardly any gravity'' \cite{Queinnec1994}. The more disciplined approach to
reflection that \rad{} takes does not possess the semantic fragility of modifying
meta-meta-intepreters.

The techniques that have been developed for reducing the interpretive overhead
of such towers of interpreters, however, still
apply (see, for example, \cite{Amin2017}, \cite{Asai2014}), as do simpler
partial-evaluation-based approaches such as the one described in \cite{Brown2017}.

\subsection{Immutable values and refs}
In \rad{} all values are immutable: once created they cannot be modified in any
way. This follows in the footsteps of functional programming
languages such as Haskell, Clojure, and Erlang.

Immutable values allow for a more \textit{local} reasoning of the
behaviour of programs; values are guaranteed not to change as a result of
calling a function. This is particularly important in the context of a
massively collaborative program where developers cannot rely on a global
view, and the program is maintained by knowledge-dissemination via code reviews
and standardized practices.

However state-machines are inherently stateful, so a reference type is included
in the language. Reading or modifying the reference must be done explicitly,
with the primops \texttt{read-ref} and \texttt{write-ref}, respectively.

\subsection{Effects} In order to maintain determinism and safety in on-chain
computations, the primitives available on-chain, $\rho$, are pure. \rad{} also
comes with an additional set of primitives, $\rho_{!}$, intended for off-chain
computations, such as scripting interactions with RSMs (these primitives are,
by convention, textually distinguished by identifiers ending with an exclamation
mark).

Expressions in the RSM therefore cannot have
side-effects; instead, they may evaluate to a value that \emph{describes} an effect,
but the decision of whether or how to actually carry on that effect happens at an
effect-handling layer. This architecture, with a central authority
administering effects received via messages, resembles work following
\cite{Cartwright1994} and \cite{Bauer2013}.

The value that results from evaluating an expression purely may, in addition to
\emph{describing} an effect, pass a continuation with the result of that
effect (if any). The effect-handling layer can chose to only call continuations
without arguments. In turn, this means the continuation will not have access to
the result of effectful computations, though it may (at the discretion of the
effect-handling layers) have output effects.

\subsection{Semantics}
\label{semantics-spec}

\rad{}'s semantics are defined in terms of an abstract state machine, that is, a
set of possible states $S$, an initial state $s_0 \in S$, a set of possible
inputs $V$, and a transition function $\S \colon S \times V \to S$.
For the precise definitions of the sets in use, see Appendix
\ref{value-definition}.

The particularity of \rad{} is that it defines a state machine whose behaviour
can change in response to some input. For this reason, \rad{}'s semantics is
defined in two steps: first we define the \emph{base semantics} $\B$, which is
the semantics of the `underlying' programming language, and then the
state-machine semantics $\S$, which specifies the particular way in which the
base semantics are invoked on each item of the input stream. Both will be
defined in terms of endomaps of the \emph{state} $S$ of the machine, associated
with each possible input, which are just elements of $V$, the set of
\emph{values}. Note that technically any value is permitted as an input, but in
practice when deployed on a network the inputs will be deserialised from some
transmission format which will most likely exclude values such as closures.

The state is composed of the \emph{environment} $E$, which associates identifiers to
values, and the \emph{memory} $M$ which tracks the values of refs:
\[
S := E \times M
\]
with
\[
E := \Ident \to 1 + V \quad \text{and} \quad M := A \to 1 + V,
\]
the (implementation specific) set $\Ident$ describes the valid identifiers. The
set $A = \mathbb{N} + \{\evalAddr\}$ contains the memory addresses used for
reference cells (a memory address is either a natural number or the
special address $\evalAddr$ which is used for storing the evaluation function).
The set of values is defined in Appendix \ref{value-definition}.

We shall adopt the following convention to make the definitions more readable:
many of the functions have as codomain a set $1 + X$ (for some set $X$), with
the left summand indicating an error. The unique element of $1$ is denoted
\texttt{error}. When defining such a function in terms of another, the errors
propagate naturally, so these cases will be supressed from the definitions.

\subsubsection{Base semantics $\B$}

The base semantics is defined as a function
\[
\B \colon S \times V \to 1 + S \times V.
\]
\begin{itemize}
\item Sequences: If $p_1, \ldots, p_n \in V$, then
  \[
    \B(s, (\mathtt{do} \ p_1 \ldots p_n)) := \B^+(s, (p_i)).
  \]
  See below for the definition of the sequence semantics $\B^+$, which defines
  the semantics over a finite sequence of inputs (not to be confused with the
  state machine semantics $\S$, which comes later).
\item Identifiers: If $i \in \Ident$ then
  \[
    B((e,m), i) := ((e,m),e(i)),
  \]
  unless $e(i) = \mathtt{error}$ in which case $B((e,m), i) = \mathtt{error}$.
\item Conditionals: For $c,t,f \in V$,
  \[
    \B( s, (\mathtt{if} \ c \ t \ f)) :=
    \begin{cases}
      \B(s', t) & \text{if $c' \neq \texttt{\#}\mathtt{f}$,}\\
      \B(s', f) & \text{otherwise,}
    \end{cases}
  \]
  where $\B(s, c) = (s', c')$.

  For $n \geq 0$, $c_i, x_i \in V$, $1 \leq i \leq n$,
  \begin{multline*}
    \B(s, (\mathtt{cond} \ c_1 \ x_1 \ldots c_n \ x_n)) :=\\
    \begin{cases}
      \mathtt{error} & \text{if $n = 0$,}\\
      \B(s', x_1) & \text{if $v \neq \texttt{\#}\mathtt{f}$,}\\
      \B(s', (\mathtt{cond} \ c_2 \ x_2 \ldots c_n \ x_n)) & \text{otherwise}.
    \end{cases}
  \end{multline*}
  where $\B(s, c_1) = (s', v)$.
\item
  \def\defe{\mathrm{def}}
  \def\calle{\mathrm{call}}
  \def\rete{\mathrm{rete}}
  Lambdas: If $(x_1 \ldots x_n) \in I^*$ and $(p_1 \ldots p_m) \in V^*$, then
  \begin{multline*}
    \B((e_\defe, m_\defe), (\mathtt{fn} \ (x_1 \ldots x_n) \ p_1 \ldots p_m)) :=\\
    ((e_\defe,m_\defe), f)
  \end{multline*}
  where $f$ is the function:
  \begin{multline*}
    ((e_\calle,m_\calle), (v_1 \ldots v_{n'})) \mapsto\\
    \begin{cases}
      \B^+((e_\defe[x_i \mapsto v_i], m_\calle), (p_i)) & \text{if $n = n'$,}\\
      \mathtt{error} & \text{otherwise.}
    \end{cases}
  \end{multline*}
\item Definitions: When $x \in \Ident$ and $p \in V$,
\[
\B((e,m), (\mathtt{def} \ x \ p)) := ((e'[x \mapsto v], m'), ())
\]
where $\B((e,m), p) = ((e', m'), v)$.
\item Applications: When $p \in V$ and $(q_1 \ldots q_n) \in V^*$, then
\[
\B(s, (p \ q_1 \ldots q_n)) := f(s_n, (v_1 \ldots v_n))
\]
where
\begin{align*}
  \B(s, p) &= (s_0, f),\\
  \B(s_0, q_1) &= (s_1, v_1),\\
  \ldots\\
  \B(s_{n-1}, q_n) &= (s_n, v_n)
\end{align*}
and $f$ is a function. If any of these result in an error, or $f$ is not a
function, then the whole computation results in an error.
\item All other inputs yield an error.
\end{itemize}

The auxiliary sequence semantics $\B^+$ is defined on sequences of values $V^*$, as follows:
  \[
    \B^+(s, (p_i)_{1 \leq i \leq n}) =
    \begin{cases}
      \mathtt{error} & \text{if $n = 0$,}\\
      \B(s, p_1) & \text{if $n = 1$,}\\
      \B^+(s', (p_i)_{2 \leq i \leq n}) & \text{otherwise,}
    \end{cases}
  \]
  where $\B(s', p_1) = (s', v')$.

\subsubsection{State-machine semantics}
The \rad{} state machine is given by an \emph{initial state} $s_0$ and a
transition function
\[
  \S \colon S \times V \to S.
\]

The initial state is defined to be $(e_0,m_0)$ where $e_0$ is the \emph{initial
  environment} (see Appendix \ref{initial-env}), and
\[
  m_0 := (\iota_L \circ !_A)[\text{\texttt{eval-addr}} \mapsto \B]
\]
is the memory which associates a value to the special address
\texttt{eval-addr}; a function which evaluates values according to the base
evaluation.

Given a value $p \in V$ and a state $(e,m) \in S$, if
$m(\text{\texttt{eval-addr}})$ is undefined, or not a function, then the machine
crashes. Otherwise
\[
  \S((e,m), p) = s'
\]
where $m(\text{\texttt{eval-addr}})((e,m), p) = (s', v')$.
That is, evaluation proceeds as specified by the original base evaluation, or
any new evaluation that has been set in the special ref with address $\evalAddr$.

\section{Sample Programs and Chains}
\label{s:examples}

In this section, we develop a better sense for the language and its
applications by considering sample programs.

\subsection{Self-amending key-value store}

In Section \ref{s:language}, we defined a simple key-value store language. It
was not, however, an \emph{amendable} one---once defined, it was impossible to
change the semantics of the running system. For short-lived chains with a narrow
purpose, this might be satisfactory. For long-lived chains, the participants
ideas on the purpose of the chain may morph over time. Forking to a new chain is
an option, but this is not ideal for two reasons:
\begin{itemize}
  \item Consensus on the purpose and semantics of the new chain must be achieved
    ``of chain''.
  \item Participants must agree on the process and logistics of migrating to a
    new chain, how to decide from which block this takes place, etc.
\end{itemize}
In radicle this can all take place in-chain, by updating the eval function.
\input{out/kv1.rad-tex}
Note that in this case, the new evaluation function that is instantiated as a
result of a an \texttt{update} command, is the result of evaluating an
expression with the original \texttt{eval} function, because of the hyperstatic
environments. Thus, in this case, no tower of interpreters is formed. Instead,
we are swapping out one eval for another.

\subsection{Currency}

TODO: discuss how we can insure that only bob can submit the command
\texttt{(transfer "bob" x n)}.

In this section we define a currency. This example demonstrates both higher
order functions and data-hiding. Indeed the \texttt{create-currency} function
defines all the state needed for the operation of the currency but then only
returns the relevant evaluation function, making the internals of the currency
unavailable to the caller. Furthermore, by returning a potential evaluation
function (rather than setting it directly), this function may be used as a
sub-behaviour of a more featurefull RSM.
\input{out/currency.rad-tex}
After loading this code, we can imagine the following \rad{} session:
\bigskip
\begin{Verbatim}[fontsize=\small]
> (write-ref eval-ref (create-currency))
=> ()
> (new-account "alice")
=> :ok
> (new-account "bob")
=> :ok
> (transfer "alice" "bob" 3)
=> :ok
> (balance "alice")
=> 7
\end{Verbatim}

\subsection{Updatable state-machine}

Most chains will operate simple state machines with state $S$, inputs $I$ and
transitions governed by a function
\[
t \colon S \times I \to S.
\]

During the lifetime of the chain, participants may decide to modify the
state-machine in some way. The target state machine they have in mind is
specified by states $S'$, inputs $I'$ and transition function $t' \colon S'
\times I' \to S'$. In order for the old-state to be preserved they must also
choose a way to map the old state into the new format, with a function $f \colon
S \to S'$.

In order to achieve this, we define a chain that accepts two sorts of inputs:
\emph{basic inputs}:
\[
\mathtt{(basic} \ s), \quad s \in S
\]
and \emph{meta-inputs}:
\[
\mathtt{(meta} \ s), \quad s \in S
\]
which are messages the participants use to coordinate in choosing a new
transition function. All basic-inputs are simply handled by $t$ to update the
current state. The meta-inputs $\bar I$ are handled by a function
\[
m \colon \bar S \times S \times \bar I \to \bar S + V
\]
which modifies the meta-state $\bar S$ of the chain, while also having read-only
access to the state $S$. When $m$ produces a value in the right summand, this is
an expression which is evaluated to a dictionary containing $f$, $t'$ and
possibly $m'$. If only $f$ and $t'$ are present, the state is translated to the
new format with $f$, and the new state machine operated by $t'$ starts accepting
new basic inputs. If $m'$ is also provided, then the protocol for redefinition
of the state-machine is also updated.

Using this higher-order function we can now demonstrate several interesting
examples of chains:
\begin{itemize}
\item Issue chain, only person who opened issue and maintainers can close, only
  maintainers can update the transition function.
\item Ultimate example: chain managing a repo. The set of maintainers is
  maintained in a config file in master branch. only maintainers can accept PRs
  to master. Only maintainers can vote on new transition functions. This means
  that the function $m$ reads not only from the vote state $\bar S$ but also the
  basic state $S$ (i.e. the source code under version control).
\item etc.
\end{itemize}

\section{Conclusion}

\subsection{Related work}

\begin{itemize}
  \item Tezos
  \item ...
\end{itemize}

\subsection{Future work}

\rad's hyperstatic environment solves one problem but brings with it another:
arbitrarily modifying the behaviour of previously defined functions is now
impossible. If all the behaviours that a likely to be modified in the future of
a chain are stored as functions behind refs, then those behaviours can be
updated during the lifetime of the chain using \texttt{write-ref}, and accessed
in subsequent definitions using \texttt{read-ref}. However it is likely that the
full range of desired modifications is not known at the time the chain is
created. In that case, if a core function needs to be updated, then
redefinitions of all the functions that depend on it also need to be submitted.
This wholesale redefinition of most of the environment is computationally
expensive and wasteful. In the future we would like to explore ways to mitigate
this issue; possibly allowing environment modifications as a effect.

Because \rad{} is used in a consensus environment, it is highly desirable that
the semantics of a chain are \emph{correct}, that is, the chain behaves in the
way the author of the code intended, and the participants (and readers of the
code) expect.

\begin{itemize}
\item A type-system helps get rid of a large class of bugs (unexpected
  behaviour), and thus promote correctness. In the future we will explore if a
  type-system can be added to \rad{} while still allowing for eval-redefinition.
\item Most type-systems won't prevent serious problems such as security issues,
  and those that would are likely to be inconvenient for low-risk chains.
  Therefore we will also investigate developing other sorts of tooling around
  \rad{} for verifying correctness. One example we have in mind is a
  QuickCheck-like tool in which one specifies a schema for valid inputs to a
  chain, and some properties which should be maintained. The tool would then
  generate many simulations in an attempt to find a minimal counter-example to
  the stated properties.
\end{itemize}

\bibliographystyle{plain}
\bibliography{bibliography,additional}
\end{multicols}

\appendix
\section{Definition of \rad{} values}
\label{value-definition}

The set of values manipulated by the \rad interpreter is defined as a coproduct
of other sets (symbols, keywords, functions etc.). The definitions rely on two
other sets being defined: $\Ident$, the set of valid identifiers, and $\String$,
the set of valid strings. These can be considered implementation dependent, but
the definitions for the OSCoin network can be found here: <formal spec website>.

TODO: formal spec website address.

\begin{align*}
  V &:= \Sym + \Keyword + \String + \Bool + \Num + \Func + \List + \Dict + \Ref\\
  \overline{V} &:= \Sym + \Keyword + \String + \Bool + \Num + \overline{\List} + \overline{\Dict}\\
  \List &:= V^*\\
  \overline{\List} &:= \overline{V}^*\\
  \Dict &:= \overline{V} \to 1 + V\\
  \overline{\Dict} &:= \overline{V} \to 1 + \overline{V}\\
  \Func &:= S \times V^* \to S \times V\\
  \Sym &:= I\\
  \Keyword &:= I\\
  \Bool &:= \{\#t, \#f\}\\
  \Ref &:= \mathbb{N}\\
  \Num &:= \mathbb{Q}
\end{align*}
The canonical injections of $\Sym, \Keyword, \String, \Bool, \Num, \Func, \List,
\Dict$ and $\Ref$ into $V$ are denoted by $\iatom, \ikeyword, \ibool, \ifunc,
\ilist, \idict$ and $\iref$ respectively. However we shall often suppress these
from the notation.

\section{Definition of \rad's initial environment}
\label{initial-env}

The initial environment is filled with a few functions that deal with
manipulating the core data-structures.

\subsection{Pure functions}

\emph{Pure functions} are those which have no effect on the state. Thus when we
say $\mathtt{f}$ is defined as the pure function $f$, we mean that in $V$ it is
represented by:
\[
  (s, v) \mapsto (s, f(v)).
\]
In this section we only define pure functions.

\begin{itemize}
\item \texttt{eq?} tests values for equality:
  \[
    (v_1, v_2) \mapsto
    \begin{cases}
      \mathtt{t}\# & \text{if } v_1 \sim v_2,\\
      \mathtt{f}\# & \text{otherwise.}
    \end{cases}
  \]
  Where the relation $\sim$ is the least equivalence relation on values $V$ such
  that
  \begin{itemize}
  \item
    it restricts to equality on the coproduct components $\Sym, \Keyword, \String,
    \Bool, \Num$ and $\Ref$,
  \item
    $\forall (x_1 \ldots x_n), (y_1 \ldots y_n) \in V^*, \ilist((x_1 \ldots
    x_n)) \sim \ilist((x_1 \ldots x_n)) \Leftrightarrow x_1 \sim y_1, \ldots, x_n
    \sim y_n$
  \item
    $\forall d_1, d_2 \in (V \to 1 + V), \idict(d_1) \sim \idict(d_2)
    \Leftrightarrow \forall x \in V, d_1(x) \sim d_2(x).$
  \end{itemize}
\item
  \texttt{empty-list} returns the empty-list.
  \texttt{cons} adds elements to the front of lists:
  \[
    (v, (v_1 \ldots v_n)) \mapsto (v v_1 \ldots v_n).
  \]
\item
  \texttt{head} and \texttt{tail} deconstruct lists, they correspond to the pure
  functions:
  \begin{align*}
    (v_1 \ldots v_n) &\mapsto
    \begin{cases}
      v_1 & \text{if } n \geq 1,\\
      \mathtt{error} & \text{otherwise}
    \end{cases},\\
    (v_1 \ldots v_n) &\mapsto
    \begin{cases}
      (v_2 \ldots v_n) & \text{if } n \geq 1,\\
      \mathtt{error} & \text{otherwise.}
    \end{cases}
  \end{align*}
\item \texttt{empty-dict}, \texttt{lookup} and \texttt{insert} allow the
  creation and querying of dicts. They correspond to the pure functions:
  \begin{align*}
    () &\mapsto (k \mapsto \mathtt{error}),\\
    (k, d) &\mapsto d(k),\\
    (k, v, d) &\mapsto \left(x \mapsto
                \begin{cases}
                  v & \text{if $x \sim k$,}\\
                  d(x) & \text{otherwise.}
                \end{cases}
                \right)
  \end{align*}
\item The pure function \texttt{string-append} concatenates a list of strings
  (implementation specific).
\item The pure functions $\mathtt{+}, \mathtt{*}, \mathtt{-}, \mathtt{<},
  \mathtt{>}$ correspond to the mathematical functions $+, \times, -, <, >$
  respectively.
\end{itemize}

\subsection{Impure functions}

There are only three impure functions and they deal with reference cells.
\begin{itemize}
\item \texttt{ref} creates a new reference cell and returns it:
  \[
    ((e,m), v) \mapsto ((e,m[a \mapsto v]), \iref(a))
  \]
  where $a$ is a new memory address, that is, $m(a) = \mathtt{error}$. How $a$
  is generated is implementation specific.
\item \texttt{read-ref!} dereferences a reference cell:
  \[
    ((e,m), r) \mapsto
    \begin{cases}
      ((e,m), m(a)) & \text{if $r = \iref(a)$ for some address $a$,}\\
      \mathtt{error} & \text{otherwise.}
    \end{cases}
  \]
\item \texttt{write-ref!} writes a new value to a reference cell:
  \[
    ((e,m), r, v) \mapsto
    \begin{cases}
      ((e,m[a \mapsto v]), ()) & \text{if $r = \iref(a)$ for some address $a$,}\\
      \mathtt{error} & \text{otherwise.}
    \end{cases}
  \]
\item Lastly, \texttt{eval-ref} holds the evaluation function, that is the
  initial environment associates the symbol \texttt{eval-ref} to the value
  $\iref(\evalAddr)$, where $\evalAddr \in A$ is the special address for the
  evaluation function. We will usually define:
  \begin{lstlisting}
    (define eval
      (lambda (expr)
        ((read-ref! eval-ref) expr)))
  \end{lstlisting}
  so that the function \texttt{eval} evaluates expressions with the current evaluator.
\end{itemize}


\end{document}
