\section{Definition of \rad{} values}
\label{value-definition}

The set of values manipulated by the \rad{} interpreter is defined as a coproduct
of other sets (symbols, keywords, functions etc.). The definitions rely on two
other sets being defined: $\Ident$, the set of valid identifiers, and $\String$,
the set of valid strings.

\begin{align*}
  V &:= \Sym + \Keyword + \String + \Bool + \Num + \Func + \List + \Vect + \Dict + \Ref\\
  \overline{V} &:= \Sym + \Keyword + \String + \Bool + \Num + \overline{\List} + \overline{\Vect} + \overline{\Dict}\\
  \List &:= V^*\\
  \overline{\List} &:= \overline{V}^*\\
  \Vect &:= V^*\\
  \overline{\Vect} &:= \overline{V}^*\\
  \Dict &:= \overline{V} \to 1 + V\\
  \overline{\Dict} &:= \overline{V} \to 1 + \overline{V}\\
  \Func &:= S \times V^* \to S \times V\\
  \Sym &:= I\\
  \Keyword &:= I\\
  \Bool &:= \{\#t, \#f\}\\
  \Ref &:= \mathbb{N}\\
  \Num &:= \mathbb{Q}
\end{align*}
The canonical injections of $\Sym$, $\Keyword$, $\String$, $\Bool$, $\Num$,
$\Func$, $\List$, $\Vect$, $\Dict$ and $\Ref$ into $V$ are denoted by $\iatom$,
$\ikeyword$, $\ibool$, $\ifunc$, $\ilist$, $\ivect$, $\idict$ and $\iref$ respectively.
However we shall often suppress these from the notation.

Note that there are two sorts of values representing sequences: lists and
vectors. While this is not a formal requirement, implementations are encouraged
to implement values of $\List$ as linked lists, and those of $\Vect$ as
data-structures with efficient access, insertion and deletion at arbitrary
indexes.

Notation:
\begin{itemize}
  \item If $(v_1, \ldots, v_n) \in V^*$ then the corresponding element of
    $\List$ is denoted by $(v_1 \ \ldots \ v_n)$, and similarly for
    $\overline{\List}$.
  \item If $(v_1, \ldots, v_n) \in V^*$ then the corresponding element of
    $\Vect$ is denoted by $[v_1 \ \ldots \ v_n]$, and similarly for $\overline{\Vect}$.
\end{itemize}
