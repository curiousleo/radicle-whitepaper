\section{Definition of \rad values}
\label{value-definition}

The set of values manipulated by the \rad interpreter is defined as a coproduct
of other sets (symbols, keywords, functions etc.). The definitions rely on two
other sets being defined: $\Ident$, the set of valid identifiers, and $\String$,
the set of valid strings. These can be considered implementation dependent, but
the definitions for the OSCoin network can be found here: <formal spec website>.

TODO: formal spec website address.

\begin{align*}
  V &:= \Sym + \Keyword + \String + \Bool + \Num + \Func + \List + \Dict + \Ref\\
  \overline{V} &:= \Sym + \Keyword + \String + \Bool + \Num + \overline{\List} + \overline{\Dict}\\
  \List &:= V^*\\
  \overline{\List} &:= \overline{V}^*\\
  \Dict &:= \overline{V} \to 1 + V\\
  \overline{\Dict} &:= \overline{V} \to 1 + \overline{V}\\
  \Func &:= V \times S \to S \times V\\
  \Sym &:= I\\
  \Keyword &:= I\\
  \Bool &:= \{\#t, \#f\}\\
  \Ref &:= \mathbb{N}\\
  \Num &:= \mathbb{Q}
\end{align*}
The canonical injections of $\Sym, \Keyword, \String, \Bool, \Num, \Func, \List,
\Dict$ and $\Ref$ into $V$ are denoted by $\iatom, \ikeyword, \ibool, \ifunc,
\ilist, \idict$ and $\iref$ respectively. However we shall often suppress these
from the notation.