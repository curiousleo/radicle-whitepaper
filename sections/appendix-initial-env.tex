\section{Definition of \rad's initial environment}
\label{initial-env}

The initial environment is filled with a few functions that deal with
manipulating the core data-structures.

\subsection{Pure functions}

\emph{Pure functions} are those which have no effect on the state. Thus when we
say $\mathtt{f}$ is defined as the pure function $f$, we mean that in $V$ it is
represented by:
\[
  (s, (v_i)_i) \mapsto (s, f((v_i)_i)).
\]
In this section we only define pure functions. Furthermore, these functions will
have a fixed arity. Calling a function with an argment of a different length
than that in the definition results in an error.

\begin{itemize}
\item \texttt{eq?} tests values for equality:
  \[
    (v_1, v_2) \mapsto
    \begin{cases}
      \mathtt{t}\# & \text{if } v_1 \sim v_2,\\
      \mathtt{f}\# & \text{otherwise.}
    \end{cases}
  \]
  Where the relation $\sim$ is the least equivalence relation on values $V$ such
  that
  \begin{itemize}
  \item
    it restricts to equality on the coproduct components $\Sym, \Keyword, \String,
    \Bool, \Num$ and $\Ref$,
  \item
    $\forall (x_1, \ldots, x_n), (y_1, \ldots, y_n) \in V^*, \ilist((x_1, \ldots,
    x_n)) \sim \ilist((x_1, \ldots, x_n)) \Leftrightarrow x_1 \sim y_1, \ldots, x_n
    \sim y_n$
  \item
    $\forall (x_1, \ldots, x_n), (y_1, \ldots, y_n) \in V^*, \ivect((x_1, \ldots,
    x_n)) \sim \ivect((x_1, \ldots, x_n)) \Leftrightarrow x_1 \sim y_1, \ldots, x_n
    \sim y_n$
  \item
    $\forall d_1, d_2 \in (\bar V \to 1 + V), \idict(d_1) \sim \idict(d_2)
    \Leftrightarrow \forall x \in \bar V, d_1(x) \sim d_2(x).$
  \end{itemize}
\item
  \texttt{empty-list} returns the empty-list.
  \texttt{cons} adds elements to the front of lists:
  \[
    (v, (v_1 \ldots v_n)) \mapsto (v \ v_1 \ldots v_n).
  \]
  and results in an error if called on a non-list value.
\item
  \texttt{head} and \texttt{tail} deconstruct lists, they correspond to the pure
  functions:
  \begin{align*}
    (v_1 \ldots v_n) &\mapsto
    \begin{cases}
      v_1 & \text{if } n \geq 1,\\
      \mathtt{error} & \text{otherwise}
    \end{cases},\\
    (v_1 \ldots v_n) &\mapsto
    \begin{cases}
      (v_2 \ldots v_n) & \text{if } n \geq 1,\\
      \mathtt{error} & \text{otherwise.}
    \end{cases}
  \end{align*}
  and result in error if called on non-lists.
\item \texttt{empty-dict}, \texttt{lookup} and \texttt{insert} allow the
  creation and querying of dicts. They correspond to the pure functions:
  \begin{align*}
    () &\mapsto (k \mapsto \mathtt{error}),\\
    (k, d) &\mapsto
             \begin{cases}
               d(k) & \text{if $k \in \bar V$ and $d \in \Dict$,}\\
               \mathtt{error} & \text{otherwise,}
             \end{cases}\\
    (k, v, d) &\mapsto \begin{cases}
                       d[k \mapsto v] & \text{if $k \in \bar V$ and $d \in \Dict$,}\\
                       \mathtt{error} & \text{otherwise.}
                       \end{cases}
  \end{align*}
\item The pure function \texttt{nth} extracts elements from lists or vectors:
  \[
    (i, v) \mapsto
    \begin{cases}
      v_i & \text{if $v$ is the list $(v_1 \ldots v_n)$ and $0 \leq i \leq n$ is
        a whole number,}\\
      v_i & \text{if $v$ is the vector $[v_1 \ldots v_n]$ and $0 \leq i \leq n$ is
      a whole number,}\\
      \mathtt{error} & \text{otherwise}
    \end{cases}
  \]
\item The pure function \texttt{string-append} concatenates a list of strings
  (implementation specific).
\item The pure functions $\mathtt{+}, \mathtt{*}, \mathtt{-}, \mathtt{<},
  \mathtt{>}$ correspond to the mathematical functions $+, \times, -, <, >$
  respectively. If called on non-numbers they result in errors.
\end{itemize}

\subsection{Impure functions}

There are only three impure functions and they deal with reference cells.
\begin{itemize}
\item \texttt{ref} creates a new reference cell and returns it:
  \[
    ((e,m), v) \mapsto ((e,m[a \mapsto v]), \iref(a))
  \]
  where $a$ is a new memory address, that is, $m(a) = \mathtt{error}$. How $a$
  is generated is implementation specific.
\item \texttt{read-ref!} dereferences a reference cell:
  \[
    ((e,m), r) \mapsto
    \begin{cases}
      ((e,m), m(a)) & \text{if $r = \iref(a)$ for some address $a$,}\\
      \mathtt{error} & \text{otherwise.}
    \end{cases}
  \]
\item \texttt{write-ref!} writes a new value to a reference cell:
  \[
    ((e,m), r, v) \mapsto
    \begin{cases}
      ((e,m[a \mapsto v]), ()) & \text{if $r = \iref(a)$ for some address $a$,}\\
      \mathtt{error} & \text{otherwise.}
    \end{cases}
  \]
\item Lastly, \texttt{eval-ref} holds the evaluation function, that is the
  initial environment associates the symbol \texttt{eval-ref} to the value
  $\iref(\evalAddr)$, where $\evalAddr \in A$ is the special address for the
  evaluation function. We will usually define:
  \begin{lstlisting}
    (def eval
      (fn [expr]
        ((read-ref! eval-ref) expr)))
  \end{lstlisting}
  so that the function \texttt{eval} evaluates expressions with the current evaluator.
\end{itemize}
