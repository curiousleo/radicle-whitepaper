\section{Language Syntax and Semantics}
\label{s:language}

While the individual parts of \rad{} do not bring anything
new to programming language theory, its intended domain of use
differs significantly from that of other programming languages and the set of
design choices that characterize \rad{} make for a unique language.

\rad{} provides a way to program the behaviour of RSMs. A particular instance of
a \rad{} RSM is an append-only sequence of \rad{} inputs, which grows over time,
and is managed by some form of \emph{consensus} (Paxos, Raft, proof-of-work,
etc.), which is valid according to the \rad{} semantics (see section
\ref{semantics-spec}), that is, the inputs are accepted, in order and without
error. We'll refer to such an RSM as a \emph{\rad{} RSM}, but because we usually
think of \rad{} as being used on a proof-of-work blockchain, or e.g. on a
Scuttlebutt feed (unforgeable, single-owner, append-only logs), we'll also refer
to such RSMs simply as \emph{chains}. In such a context it is still useful to
run some computations through the \rad{} interpreter locally, that is, evaluate
some inputs speculatively without submitting them the underlying consensus
mechanism. Such computations are referred to as \emph{off-chain}, while those of
the RSM that are under consensus are referred to as \emph{on-chain}.

\rad{} strives:
\begin{itemize}
\item[(1)] To be deterministic, so that \rad{} programs specifiy deterministic
  state machines.
\item[(2)] To be powerful enough to be a `universal state-machine'.
\item[(3)] To be able to restrict this power appropriately, so that malicious
  inputs can be rejected, and to aid reasoning about the behaviour of valid
  inputs.
\item[(4)] To be concise, expressive and emphasize correctness, so that new
  chains, with new semantics can be created cheaply (in terms of development
  time), be easily understood by all participants, and hopefully bug-free.
\item[(5)] To work well in a collaborative setting, in which code is submitted
  by multiple parties, in a potentially unspecified order.
% TODO: point (5) is unconvincing
\item[(6)] To have the simplest possible underlying semantics, so
  that \rad{} interpreters are well specified, and so that \rad{} can be used in
  security-sensitive situations.
\item[(7)] To have the ability to interpret itself, so that the semantics of a
  chain may be modified according to its own semantics, that is, on-chain.
\end{itemize}

The design choices we made with \rad{} are:
\begin{itemize}
\item[(a)] It is a high-level, homoiconic LISP dialect.
\item[(b)] It has the ability to redefine a special function \texttt{eval}, which
  is used as part of its own evaluation.
\item[(c)] It has first-class functions.
\item[(d)] It is dynamically typed.
\item[(e)] It is lexically scoped with a \emph{hyperstatic} global environment.
\item[(f)] It only has immutable values.
\item[(g)] It has a deterministic effect system for managing state (references).
\end{itemize}

\begin{figure}[H]
\small
\begin{tabular}{l|ccccccc}
    & a & b & c & d & e & f & g\\
  \hline
  1 &   &   &   &   &   &   & + \\
  2 & + &   &   &   &   &   & \\
  3 &   & +\\
  4 & + &   & + & - &   & +\\
  5 & \\
  6 & + &   &   &   & + & +
\end{tabular}
\normalsize
\caption{Summary of design choices and how they interact with \rad{}'s design goals.}
\end{figure}

LISP is well-known for being a high-level family of languages having very concise
self-interpreters, and the Scheme specification has boiled down the
semantics to a small core. Choosing a LISP helps with (2), (4) and (7), and, in
particular, basing the design on Scheme helps with (6).

The redefinition of \texttt{eval} is what enables (3).

Languages with first-class functions are particularly expressive, so (c) helps
with (4).


Point (d) is a compromise: it is hard to satisfy the other points (especially
(7)) with strong static types, even though this would help with (4). Though
adding types to a language which can interpret itself is not impossible (see
\cite{brown-papers}), this constraint would also inhibit the ability to
compile other languages into \rad{}.

The \emph{hyperstatic} global environment means that the resolution of free
variables takes place at the definition site rather than call-site. See Section
\ref{s:hyperstatic} for explanations for how this helps with (6).

The language is kept as pure and immutable as possible, which limits
expressivity but emphasizes clarity and correctness (goal (4)). The abundance of
mutability would also probably hinder (6).

Since RSMs are inherently stateful, we felt that the inclusion of a reference
type for managing state would help in the specification of semantics, however we
will discuss the alternatives.

The remainder of this section expands upon these goals and
design choices.


\subsection{LISP}

Scheme has the particularity that a minimal interpreter of itself can
be implemented in a few lines of code. This is achieved by:
\begin{itemize}
  \item Code being represented by the core data-structures of the language, so
      that manipulating code is as straighforward as any data manipulation.
  \item The language being built up by a very small set of primitives: a few
    special forms and some primitive functions.
\end{itemize}
Let's consider a basic example of a chain which might want to change a part of
its semantics. In this chain, participants may want to \texttt{boot} other
participants if some condition is met, which is defined by a predicate
\texttt{bootable?}. For example, \texttt{bootable?} might be defined as:
\begin{lstlisting}
(define bootable?
  (lambda (a b)
    (and (admin? a) (not (admin? b)))))
\end{lstlisting}
That is, admins may boot non-admins. At some point the participants may decide
that this is too crude, and that a participant should only be booted if the
majority of participants agree. For this change to take place, most likely one of
the inputs to the system will contain code for a new version of
\texttt{bootable?} (the semantics of when such a message is to be accepted would
probably be defined by yet another predicate). This code must be transmited and
interpreted in the simplest and most transparent way possible. By choosing a LISP,
this code is represented by simple data-structures, and the interpretation process is
very direct (simplifying parsing and abstract syntax trees, etc.). This
gives the participants more confidence in their understanding of the current and
proposed semantics.

\subsection{Data types}
\rad{}'s data types are: booleans (\texttt{\#t} and
\texttt{\#f}); strings (a sequence of characters within double quotes, with
\verb$\$ as the escape character), symbols, lists, dictionaries
(maps), and numbers (currently only arbitrary-precision decimals). These
datatypes are all immutable; additionally \rad{} supports \emph{refs}---mutable
references that can hold any other datatype.

For more details on radicle values, see Appendix \ref{value-definition}.

\subsection{Hyperstatic Environments}
\label{s:hyperstatic}

Most scripting languages are intended for programs where the developer can
decide exactly where code should be placed, and be fully aware of the context
that precedes it. One can always add a line \texttt{y = bar(x)} immediately
\emph{after} \texttt{x = foo(3)}, and not worry if something else was
added in between that changes what \texttt{x} refers to (excepting pathological
cases of distributed code collaboration).

This is not true in the environment in which \rad{} is intended to operate. In this
environment, users submit individual expressions or declarations to a running
system that is \emph{at the same time} also accepting inputs from other users.
Moreover, the view of the program that the submitter has at the time of
submission may \emph{already} be outdated. This aspect is even more significant
in the context of a blockchain \cite{Nakamoto2008}, where expressions are
queued until a block is created by a node, but the expressions in one queue are not
necessarily accessible to other nodes.

This fact may lead to bugs or attacks. Consider the following snippet:
\begin{lstlisting}
(define transfer (lambda (from to amount) ...))
(define account-alice ...)
(define account-bob ...)
(define transfer-to-alice (lambda (from amount)
  (transfer from account-alice amount))
\end{lstlisting}
A user now wants to transfer some amount to Alice. However, in a language such
as Scheme the function call \texttt{(transfer-to-alice account-bob 10)} may
behave differently than this snippet alone indicates in two ways. First,
\texttt{transfer-to-alice} and \texttt{account-bob}, which explicitly appear in
the expression being submitted, may be shadowed by a new definition. Second, the
variables \texttt{transfer}, \texttt{account-alice}, and \emph{any other} free
variables used in the body of \texttt{transfer} or transitively, in the function
bodies of the functions \texttt{transfer} calls, may have been redefined.

One solution is to not allow the redefinition of variables at all. This is
indeed what some smart contract \cite{Szabo94} languages do, and can be implemented easily in
\rad{} as well (see Section \ref{s:reflective-towers}). But this comes with its
own problems; one cannot define an improved version of a function that shadows
the old one, and must instead accept a more and more polluted environment (which
additionally may impact memory usage).

An alternative is to implement a mechanism (call it \texttt{protect-at-line}) to
make expressions invalid if any of the variables they rely on have been
redefined since a specified line of code.
However this option is quite severe. Consider:
\begin{lstlisting}
(define x 3)
(define foo (lambda () x))
(define x 5)
\end{lstlisting}
The function call \texttt{(protect-at-line 2 (foo))} will fail despite having a
perfectly reasonable interpretation---namely, \texttt{(foo)} \emph{with}
\texttt{x} \emph{as if used after line 2} (i.e., with \texttt{x} being
\texttt{3}). For more complex expressions, which may rely on many variables,
this problem becomes more significant.

This reasonable interpretation is in fact precisely what hyperstatic
environments provide. The function call \texttt{(foo)} in line 4 would, in this
model, still evaluate to \texttt{3}. The free variables of a function refer to
the values of variables \emph{when they were defined}. This is the semantics
\rad{} adopts in general (and can be found in a few other languages, such as
Forth and Standard ML\cite{SML1997}).

Now \texttt{(protect-at-line (foo))} may be given a much simpler definition:
\texttt{(foo)} if \texttt{foo} \emph{alone} has not been redefined, and otherwise
an exception. If \texttt{x} has been redefined, this is no longer of concern, as this
redefinition does not change the meaning of \texttt{foo}.

%%% End section

\subsection{\texttt{eval} redefinition}

\rad{} should be able to emulate any state machine. Of course, most state-machines
don't accept arbitrary inputs, and certainly not inputs which change the
semantics of the state-machine. Thus \rad{} should have some built in mechanism
for controlling how inputs are interpreted. To begin, inputs are sent
directly for evaluation by the `base' evaluator, so it is natural to associate
modifying this process with redefining a function which represents evaluation.
When taken to the extreme, \emph{all} evaluation is done with a call to the
special function \texttt{eval}, and this creates a reflective tower of
interpreters. In \rad{}, a more limited form of \texttt{eval}-redefinition is
available. As an example:
\bigskip
\begin{Verbatim}
> (write-ref eval-ref (lambda (expr) 3))
=> ()
> 5
=> 3
\end{Verbatim}
Contrast this with the sort of \texttt{eval}-redefinition that is available in
the language Black\cite{Asai1997}:
\begin{lstlisting}
(exec-at-metalevel
  (let ((old-eval base-eval))
    (set! base-eval
          (lambda (exp env cont)
            (write exp) (newline) (old-eval exp env cont)))))
\end{lstlisting}
After this the Black REPL will behave as follows:
\bigskip
\begin{Verbatim}
> (+ 1 2)
(+ 1 2)
+
1
2
=> 3
\end{Verbatim}
Thus, the new evaluation function is used recursively, at all levels of the
expression \texttt{(+ 1 2)}.

Compare this to the following \rad{} session:
\bigskip
\begin{Verbatim}
> (define old (read-ref eval-ref))
> (define new (lambda (e) (do (print! e) (old e))))
> (write-ref eval-ref new)
=> ()
> (+ 1 2)
(+ 1.0 2.0)
=> 3.0
\end{Verbatim}
In Black, \texttt{old-eval} calls out to
\texttt{base-eval} and, upon subsequent evaluation, this will refer to the new
definition. This is what makes the new evalution behaviour take effect at all
levels of evaluation. In \rad{}, the new evaluation is only invoked at the
topmost level, on new inputs. If one wants to make the evaluation behaviour
recursive this must be coded explicitely.

In most cases a \rad{} chain will define a domain-specific language to be
interpreted in some narrow way, with only some calls to the base eval for
reifying function definitions in case one wants to modify the behaviour of the
state-machine. Furthermore, this form of evaluation redefinition doesn't (by
default) create a tower of interpreters. Evaluating in a tower requires advanced
interpretation methods to avoid performance overheads, and this is all the more
important in \rad{}'s case where the chains may be very long-lived. Such methods
are still the subject of active reasearch (see for example \cite{Asai2014},
\cite{Thyer1999}, \cite{Amin2017}, \cite{Brown2017}), and may interfere with
other requirements of the language. Moreover, since eval-redefinition affects
all sub-expressions, and may affect not just the ``immediate'' interpreter, but
ones in which it is defined, reasoning about eval-redefinitions in towers can be
very difficult. For these reasons we believe that this limited form of
eval-redefinition suits the needs of \rad{} better than the degree of reflection
that can be found in languages such as Black.

%% \rad does not have macros. Instead, it allows for a complete redefinition of
%% the intepreter via redefinitions of \texttt{eval}.

%% \begin{verbatim}
%% (define eval (lambda (expr) 3)) ;;
%% 5 ;; => 3
%% \end{verbatim}

%% This enables users to very easily define sublanguages for new chains, or amend
%% the language running in the current one.

For a more involved example, consider a simplistic key-value store:
\input{out/kv0.rad-tex}
After which we have:
\bigskip
\begin{Verbatim}
> (set key1 3)
=> ()
> (get key1)
=> 3
> (+ 3 2)
=> 'invalid-command
\end{Verbatim}
Note that nested expressions are not evaluated:
\medskip
\begin{Verbatim}
> (set key2 (+ 3 2))
=> ()
> (get key2)
=> '(+ 3 2)
\end{Verbatim}

% TODO: maybe move to Related Work section
\subsection{Connection to Reflective Towers}
\label{s:reflective-towers}

The \texttt{eval}-redefinition mechanism resembles prior work on
\emph{reflective towers}. A reflective tower is an infinite sequence of
interpreters (called `levels') $L_0$, $L_1$, ..., where level $L_n$ is
interpreted by $L_{n+1}$. Reflective towers allow both \emph{reification}---the
ability to inspect a computation via constructs of a higher level---and
\emph{reflection}, that is, the ability to define and enter new, lower levels.
Conceptually, \rad{} differs from reflective towers by only allowing reflection.
Thus, the only levels that exist are the ones programs create. Queinnec has
quipped that reflective languages ``plunge us into a world with few laws, and
hardly any gravity'' \cite{Queinnec1994}. The more disciplined approach to
reflection that \rad{} takes does not possess the semantic fragility of modifying
meta-meta-intepreters.

The techniques that have been developed for reducing the interpretive overhead
of such towers of interpreters, however, still
apply (see, for example, \cite{Amin2017}, \cite{Asai2014}), as do simpler
partial-evaluation-based approaches such as the one described in \cite{Brown2017}.

\subsection{Immutable values and refs}
In \rad{} all values are immutable: once created they cannot be modified in any
way. This follows in the footsteps of functional programming
languages such as Haskell, Clojure, and Erlang.

Immutable values and allows for a more \textit{local} reasoning of the
behaviour of programs; values are guaranteed not to change as a result of
calling a function. This is particularly important in the context of a
massively collaborative program where developers cannot rely on a global
view, and the program is maintained by knowledge-dissemination via code reviews
and standardized practices.

However state-machines are inherently stateful, so a reference type is included
in the language. Reading or modifying the reference must be done explicitly,
with the primops \texttt{read-ref} and \texttt{write-ref}, respectively.

\subsection{Effects} In order to maintain determinism and safety in on-chain
computations, the primitives available on-chain, $\rho$, are pure. \rad{} also
comes with an additional set of primitives, $\rho_{!}$, intended for off-chain
computations, such as scripting interactions with RSMs (these primitives are,
by convention, textually distinguished by identifiers ending with an exclamation
mark).

Expressions in the RSM therefore cannot have
side-effects; instead, they may evaluate to a value that \emph{describes} an effect,
but the decision of whether or how to actually carry on that effect happens at an
effect-handling layer. This architecture, with a central authority
administering effects received via messages, resembles work following
\cite{Cartwright1994} and \cite{Bauer2013}.

The value that results from evaluating an expression purely may, in addition to
\emph{describing} an effect, pass a continuation with the result of that
effect (if any). The effect-handling layer can chose to only call continuations
without arguments. In turn, this means the continuation will not have access to
the result of effectful computations, though it may (at the discretion of the
effect-handling layers) have output effects.

\subsection{Semantics}
\label{semantics-spec}

\rad{}'s semantics are defined in terms of an abstract state machine, that is, a
set of possible states $S$, an initial state $s_0 \in S$, a set of possible
inputs $V$, and a transition function $\S \colon S \times V \to S$.
For the precise definitions of the sets in use, see Appendix
\ref{value-definition}.

The particularity of \rad{} is that it defines a state machine whose behaviour
can change in response to some input. For this reason, \rad{}'s semantics is
defined in two steps: first we define the \emph{base semantics} $\B$, which is
the semantics of the `underlying' programming language, and then the
state-machine semantics $\S$, which specifies the particular way in which the
base semantics are invoked on each item of the input stream. Both will be
defined in terms of endomaps of the \emph{state} $S$ of the machine, associated
with each possible input, which are just elements of $V$, the set of
\emph{values}. Note that technically any value is permitted as an input, but in
practice when deployed on a network the inputs will be deserialised from some
transmission format which will most likely exclude values such as closures.

The state is composed of the \emph{environment} $E$, which associates identifiers to
values, and the \emph{memory} $M$ which tracks the values of refs:
\[
S := E \times M
\]
with
\[
E := \Ident \to 1 + V \quad \text{and} \quad M := A \to 1 + V,
\]
the (implementation specific) set $\Ident$ describes the valid identifiers. The
set $A = \mathbb{N} + \{\evalAddr\}$ contains the memory addresses used for
reference cells (a memory address is either a natural number or the
special address $\evalAddr$ which is used for storing the evaluation function).
The set of values is defined in Appendix \ref{value-definition}.

We shall adopt the following convention to make the definitions more readable:
many of the functions have as codomain a set $1 + X$ (for some set $X$), with
the left summand indicating an error. The unique element of $1$ is denoted
\texttt{error}. When defining such a function in terms of another, the errors
propagate naturally, so these cases will be supressed from the definitions.

\subsubsection{Base semantics $\B$}

The base semantics is defined as a function
\[
\B \colon S \times V \to 1 + S \times V.
\]
\begin{itemize}
\item Sequences: If $p_1, \ldots, p_n \in V$, then
  \[
    \B(s, (\mathtt{do} \ p_1 \ldots p_n)) := \B^+(s, (p_i)).
  \]
  See below for the definition of the sequence semantics $\B^+$, which defines
  the semantics over a finite sequence of inputs (not to be confused with the
  state machine semantics $\S$, which comes later).
\item Identifiers: If $i \in \Ident$ then
  \[
    B((e,m), i) := ((e,m),e(i)),
  \]
  unless $e(i) = \mathtt{error}$ in which case $B((e,m), i) = \mathtt{error}$.
\item Conditionals: For $c,t,f \in V$,
  \[
    \B( s, (\mathtt{if} \ c \ t \ f)) :=
    \begin{cases}
      \B(s', t) & \text{if $c' \neq \mathtt{\#f}$,}\\
      \B(s', f) & \text{otherwise,}
    \end{cases}
  \]
  where $\B(s, c) = (s', c')$.

  For $n \geq 0$, $c_i, x_i \in V$, $1 \leq i \leq n$,
  \begin{multline*}
    \B(s, (\mathtt{cond} \ c_1 \ x_1 \ldots c_n \ x_n)) :=\\
    \begin{cases}
      \mathtt{error} & \text{if $n = 0$,}\\
      \B(s', x_1) & \text{if $v \neq \#\mathtt{f}$,}\\
      \B(s', (\mathtt{cond} \ c_2 \ x_2 \ldots c_n \ x_n)) & \text{otherwise}.
    \end{cases}
  \end{multline*}
  where $\B(s, c_1) = (s', v)$.
\item
  \def\defe{\mathrm{def}}
  \def\calle{\mathrm{call}}
  \def\rete{\mathrm{rete}}
  Lambdas: If $(x_1 \ldots x_n) \in I^*$ and $(p_1 \ldots p_m) \in V^*$, then
  \begin{multline*}
    \B((e_\defe, m_\defe), (\mathtt{lambda} \ (x_1 \ldots x_n) \ p_1 \ldots p_m)) :=\\
    ((e_\defe,m_\defe), f)
  \end{multline*}
  where $f$ is the function:
  \begin{multline*}
    ((e_\calle,m_\calle), (v_1 \ldots v_{n'})) \mapsto\\
    \begin{cases}
      \B^+((e_\defe[x_i \mapsto v_i], m_\calle), (p_i)) & \text{if $n = n'$,}\\
      \mathtt{error} & \text{otherwise.}
    \end{cases}
  \end{multline*}
\item Definitions: When $x \in \Ident$ and $p \in V$,
\[
\B((e,m), (\mathtt{define} \ x \ p)) := ((e'[x \mapsto v], m'), ())
\]
where $\B((e,m), p) = ((e', m'), v)$.
\item Applications: When $p \in V$ and $(q_1 \ldots q_n) \in V^*$, then
\[
\B(s, (p \ q_1 \ldots q_n)) := f(s_n, (v_1 \ldots v_n))
\]
where
\begin{align*}
  \B(s, p) &= (s_0, f),\\
  \B(s_0, q_1) &= (s_1, v_1),\\
  \ldots\\
  \B(s_{n-1}, q_n) &= (s_n, v_n)
\end{align*}
and $f$ is a function. If any of these result in an error, or $f$ is not a
function, then the whole computation results in an error.
\item All other inputs yield an error.
\end{itemize}

The auxiliary sequence semantics $\B^+$ is defined on sequences of values $V^*$, as follows:
  \[
    \B^+(s, (p_i)_{1 \leq i \leq n}) =
    \begin{cases}
      \mathtt{error} & \text{if $n = 0$,}\\
      \B(s, p_1) & \text{if $n = 1$,}\\
      \B^+(s', (p_i)_{2 \leq i \leq n}) & \text{otherwise,}
    \end{cases}
  \]
  where $\B(s', p_1) = (s', v')$.

\subsubsection{State-machine semantics}
The \rad{} state machine is given by an \emph{initial state} $s_0$ and a
transition function
\[
  \S \colon S \times V \to S.
\]

The initial state is defined to be $(e_0,m_0)$ where $e_0$ is the \emph{initial
  environment} (see Appendix \ref{initial-env}), and
\[
  m_0 := (\iota_L \circ !_A)[\text{\texttt{eval-addr}} \mapsto \B]
\]
is the memory which associates a value to the special address
\texttt{eval-addr}; a function which evaluates values according to the base
evaluation.

Given a value $p \in V$ and a state $(e,m) \in S$, if
$m(\text{\texttt{eval-addr}})$ is undefined, or not a function, then the machine
crashes. Otherwise
\[
  \S((e,m), p) = s'
\]
where $m(\text{\texttt{eval-addr}})((e,m), p) = (s', v')$.
That is, evaluation proceeds as specified by the original base evaluation, or
any new evaluation that has been set in the special ref with address $\evalAddr$.
