\section{Sample Programs and Chains}
\label{s:examples}

In this section, we develop a better sense for the language and its
applications by considering sample programs.

\subsection{Self-amending key-value store}

In Section \ref{s:language}, we defined a simple key-value store language. It
was not, however, an \emph{amendable} one---once defined, it was impossible to
change the semantics of the running system. For short-lived chains with a narrow
purpose, this might be satisfactory. However, for long-lived chains, participants'
ideas as to the purpose of the chain may change over time. Forking to a new chain is
an option, but this is not ideal for two reasons:
\begin{itemize}
  \item Consensus on the purpose and semantics of the new chain must be achieved
    ``off chain''.
  \item Participants must agree on the process and logistics of migrating to a
    new chain, how to decide from which block this takes place, etc.
\end{itemize}
In radicle this can all take place on-chain, by updating the eval function.
\input{out/kv1.rad-tex}
Note that in this case, the new evaluation function that is instantiated as a
result of an \texttt{update} command, is the result of evaluating an
expression with the original \texttt{eval} function, because of the hyperstatic
environment. Thus, in this case, no tower of interpreters is formed. Instead,
we are swapping out one eval for another.

\subsection{Currency}

TODO: discuss how we can insure that only bob can submit the command
\texttt{(transfer "bob" x n)}.

In this section we define a currency. This example demonstrates both higher
order functions and data-hiding. Indeed the \texttt{create-currency} function
defines all the state needed for the operation of the currency but then only
returns the relevant evaluation function, making the internals of the currency
unavailable to the caller. Furthermore, by returning a potential evaluation
function (rather than setting it directly), this function may be used as a
sub-behaviour of a more featurefull RSM.
\input{out/currency.rad-tex}
After loading this code, we can imagine the following \rad{} session:
\bigskip
\begin{Verbatim}[fontsize=\small]
> (write-ref eval-ref (create-currency))
=> ()
> (new-account "alice")
=> :ok
> (new-account "bob")
=> :ok
> (transfer "alice" "bob" 3)
=> :ok
> (balance "alice")
=> 7
\end{Verbatim}

\subsection{Updatable state-machine}

Most chains will operate simple state machines with state $S$, inputs $I$ and
transitions governed by a function
\[
t \colon S \times I \to S.
\]

During the lifetime of the chain, participants may decide to modify the
state-machine in some way. The target state machine they have in mind is
specified by states $S'$, inputs $I'$ and transition function $t' \colon S'
\times I' \to S'$. In order for the old-state to be preserved they must also
choose a way to map the old state into the new format, with a function $f \colon
S \to S'$.

In order to achieve this, we define a chain that accepts two sorts of inputs:
\emph{basic inputs}:
\[
\mathtt{(basic} \ s), \quad s \in S
\]
and \emph{meta-inputs}:
\[
\mathtt{(meta} \ s), \quad s \in S
\]
which are messages the participants use to coordinate in choosing a new
transition function. All basic-inputs are simply handled by $t$ to update the
current state. The meta-inputs $\bar I$ are handled by a function
\[
m \colon \bar S \times S \times \bar I \to \bar S + V
\]
which modifies the meta-state $\bar S$ of the chain, while also having read-only
access to the state $S$. When $m$ produces a value in the right summand, this is
an expression which is evaluated to a dictionary containing $f$, $t'$ and
possibly $m'$. If only $f$ and $t'$ are present, the state is translated to the
new format with $f$, and the new state machine operated by $t'$ starts accepting
new basic inputs. If $m'$ is also provided, then the protocol for redefinition
of the state-machine is also updated.

Using this higher-order function we can now demonstrate several interesting
examples of different chains:
\begin{itemize}
\item Issue chain: only the person who opened an issue and maintainers can close the
  issue, only maintainers can update the transition function.
\item Ultimate example: chain managing a repo. The set of maintainers is
  maintained in a config file in a master branch. Only maintainers can accept PRs
  to master. Only maintainers can vote on new transition functions. This means
  that the function $m$ reads not only from the vote state $\bar S$, but also the
  basic state $S$ (i.e. the source code under version control).
\end{itemize}
