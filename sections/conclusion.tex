\section{Conclusion}

\subsection{Related work}

\begin{itemize}
  \item Tezos
  \item ...
\end{itemize}

\subsection{Future work}

\rad's hyperstatic environment solves one problem but brings with it another:
arbitrarily modifying the behaviour of previously defined functions is now
impossible. If all the behaviours that a likely to be modified in the future of
a chain are stored as functions behind refs, then those behaviours can be
updated during the lifetime of the chain using \texttt{write-ref}, and accessed
in subsequent definitions using \texttt{read-ref}. However it is likely that the
full range of desired modifications is not known at the time the chain is
created. In that case, if a core function needs to be updated, then
redefinitions of all the functions that depend on it also need to be submitted.
This wholesale redefinition of most of the environment is computationally
expensive and wasteful. In the future we would like to explore ways to mitigate
this issue; possibly allowing environment modifications as a effect.

Because \rad{} is used in a consensus environment, it is highly desirable that
the semantics of a chain are \emph{correct}, that is, the chain behaves in the
way the author of the code intended, and the participants (and readers of the
code) expect.

\begin{itemize}
\item A type-system helps get rid of a large class of bugs (unexpected
  behaviour), and thus promote correctness. In the future we will explore if a
  type-system can be added to \rad{} while still allowing for eval-redefinition.
\item Most type-systems won't prevent serious problems such as security issues,
  and those that would are likely to be inconvenient for low-risk chains.
  Therefore we will also investigate developing other sorts of tooling around
  \rad{} for verifying correctness. One example we have in mind is a
  QuickCheck-like tool in which one specifies a schema for valid inputs to a
  chain, and some properties which should be maintained. The tool would then
  generate many simulations in an attempt to find a minimal counter-example to
  the stated properties.
\end{itemize}